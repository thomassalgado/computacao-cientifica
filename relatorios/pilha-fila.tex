% !TEX encoding = UTF-8 Unicode
\documentclass[rascunho,xindy,sublist]{fei}
\usepackage[utf8]{inputenc}
\usepackage{listings}
\usepackage{xcolor}
\usepackage{mdframed}
\lstset { %
    language=C++,
    backgroundcolor=\color{black!5}, % set backgroundcolor
    basicstyle=\footnotesize,% basic font setting
}

\definecolor{listinggray}{gray}{0.9}
%\definecolor{lbcolor}{rgb}{0.9,0.9,0.9}
\definecolor{lbcolor}{rgb}{1,1,1}
\lstset{
backgroundcolor=\color{lbcolor},
    	tabsize=4,    
    	language=[GNU]C++,
        basicstyle=\scriptsize,
        upquote=true,
        aboveskip={1.5\baselineskip},
        columns=fixed,
        showstringspaces=false,
        extendedchars=false,
        breaklines=true,
        prebreak = \raisebox{0ex}[0ex][0ex]{\ensuremath{\hookleftarrow}},
        frame=single,
        numbers=left,
        showtabs=false,
        showspaces=false,
        showstringspaces=false,
        identifierstyle=\ttfamily,
        keywordstyle=\color[rgb]{0,0,1},
        commentstyle=\color[rgb]{0.026,0.112,0.095},
        stringstyle=\color[rgb]{0.627,0.126,0.941},
        numberstyle=\color[rgb]{0.205, 0.142, 0.73}
}
\lstset{
  	backgroundcolor=\color{lbcolor},
	tabsize=4,
  	language=C++,
  	captionpos=b,
  	tabsize=3,
  	frame=lines,
  	numbers=left,
  	numberstyle=\tiny,
  	numbersep=5pt,
  	breaklines=true,
  	showstringspaces=false,
  	basicstyle=\footnotesize,
  	keywordstyle=\color[rgb]{0,0,1},
  	commentstyle=\color[rgb]{0,0.502,0},
  	stringstyle=\color{red}
}


\newenvironment{myenv}[1]
  {\mdfsetup{
    frametitle={\colorbox{white}{\space#1\space}},
    innertopmargin=10pt,
    frametitleaboveskip=-\ht\strutbox,
    frametitlealignment=\center
    }
  \begin{mdframed}%
  }
  {\end{mdframed}}

\author{Rafael Thomas Pontes Salgado}
\title{Relatório 01}
\subtitulo{Pilhas e Filas­}

\cidade{São Bernardo do Campo}
\instituicao{Centro Universitário FEI}

%%%% -- Entradas Listas de Abreviaturas e Simbolos
%%%%%%%%%%%%%%%%%%%%%%%%%%%%%%%%%%%%%%%%%%%%%%%%%%%%%%%%%%%%%%%%%%%%%%%%%%%%%%%%%%%%%%%%%%%%%%%%%%%%%%%%%
%%%% -- Titulos - comentar caso a respectiva lista nao seja utilizada
\newglossaryentry{acro}{name={},description={\nopostdesc},sort=a} %Usado para alinhar a lista de abreviaturas
\newglossaryentry{geral}{name={Geral},description={\nopostdesc},sort=a}
\newglossaryentry{greek}{name={Letras gregas},description={\nopostdesc},sort=b}
\newglossaryentry{sub}{name={Subscritos},description={\nopostdesc},sort=c}

%% -- Abreviaturas
\newacronym[longplural=Centro Universitário da FEI,parent=acro]{fei}{FEI}{Centro Universitário da FEI}

%% -- Simbolos
%% -- Latin letters
%\newglossaryentry{}{parent=geral,type=symbols,name={},sort=a,description={}
\newglossaryentry{A}{parent=geral,type=symbols,name={\ensuremath{A}},sort=a,description={exchanger total heat transfer area, $m^2$}}
\newglossaryentry{G}{parent=geral,type=symbols,name={\ensuremath{G}},sort=g,description={exchanger flow-stream mass velocity, $kg/(s m^2)$}}
\newglossaryentry{f}{parent=geral,type=symbols,name={\ensuremath{j}},sort=j,description={friction factor, dimensionless}}

%% -- Greek letters
%\newglossaryentry{}{parent=geral,type=symbols,name={},sort=a,description={}
\newglossaryentry{deltap}{parent=greek,type=symbols,name={\ensuremath{\Delta P}},sort=p,description={pressure drop, $Pa$}}
\newglossaryentry{nu}{parent=greek,type=symbols,name={\ensuremath{\nu}},sort=b,description={specific volume, $m^3/kg$}}
\newglossaryentry{beta}{parent=greek,type=symbols,name={\ensuremath{\beta}},sort=b,description={ratio of free-flow area $A_{ff}$ and frontal area $A_{fr}$ of one side of exchanger, dimensionless}}

%% -- Subscripts
%\newglossaryentry{}{parent=geral,type=symbols,name={},sort=a,description={}
\newglossaryentry{fr}{parent=sub,type=symbols,name={\ensuremath{fr}},sort=fr,description={frontal}}
\newglossaryentry{in}{parent=sub,type=symbols,name={\ensuremath{i}},sort=in,description={inlet}}
\newglossaryentry{out}{parent=sub,type=symbols,name={\ensuremath{o}},sort=out,description={outlet}}
%%%%%%%%%%%%%%%%%%%%%%%%%%%%%%%%%%%%%%%%%%%%%%%%%%%%%%%%%%%%%%%%%%%%%%%%%%%%%%%%%%%%%%%%%%%%%%%%%%%%%%%

\addbibresource{referencias.bib}

\makeindex
\makeglossaries

\begin{document}

\maketitle

\begin{resumo}
Nesse relatório buscamos expor a implementação de duas estruturas de dados, Fila e Pilha, mostrando como são conceitualmente, para quais problemas são utilizadas e exemplos de implementação em linguagem C++ 
\palavraschave{Pilha. Fila. Estrutura de Dados. Algoritmo. C++}
\end{resumo}

\listoffigures
\listoftables
\printglossaries
\tableofcontents

\chapter{Introdução}

Durante nossa vida nos deparamos com inúmeros problemas diários onde uma organização deve ser dada aos elementos para que possamos resolver os problemas de forma eficaz. Quando vamos ao banco pagar uma conta ou quando simplesmente vamos organizar as cartas de um baralho, certas estruturas naturais estão sempre presentes para nos auxiliar e as mais comumente encontradas são as estruturas de pilha e fila que serão o objetivo desse relatório.

\section{Motivação}

Por serem estruturas de dados utilizadas de forma recorrente em diversos algoritmos, esse trabalho tem a motivação de estudar as estruturas de dados de pilha e fila para seu maior entendimento e ajuda na escolha das mesmas em futuras implementações.

\section{Objetivo}

Esse relatório tem como objetivo a implementação das estruturas de dados de Pilha e Fila utilizando a linguagem orientada a objeto C++, verificando sua complexidade de implementação e exemplificando usos das mesmas.

\section{Metodologia}

Para a implementação das estruturas de dados de pilha e fila, foi utilizada a linguagem orientada a objeto C++ visto a fácil representação das estruturas como objetos.

\chapter{Teoria}

A seguir iremos descrever as estruturas de pilha e fila, bem como suas propriedades e métodos

\section{Pilha}

Pilha é uma estrutura de dados onde um conjunto de elementos é organizado da forma \textbf{último a entrar, primeiro a sair}, possuindo as seguintes operações:

\begin{enumerate}
  \item \textbf{\textit{Inserir}} ou \textbf{\textit{Push}}
  \item \textbf{\textit{Remover}} ou \textbf{\textit{Pop}}
  \item \textbf{\textit{Topo}} ou \textbf{\textit{Top}}
\end{enumerate}

O método \textbf{\textit{Inserir}} adiciona um novo elemento a estrutura de dados fazendo com que o mesmo seja representado como estando no topo da pilha

O metodo \textbf{\textit{Remover}} retira o elemento atualmente no top da pilha da estrutura de dados, o elemento subsequente passa a ser o novo topo da pilha ou a mesma ficará vazia.

O metodo \textbf{\textit{Topo}} informa qual é o elemento que esta atualmente no topo da pilha, sem removê-lo.

\section{Fila}

Fila é uma estrutura de dados onde um conjunto de elementos é organizado da forma \textbf{primeiro a entrar, primeiro a sair}, possuindo as seguintes operações:

\begin{enumerate}
  \item \textbf{\textit{Inserir}} ou \textbf{\textit{Enqueue}}
  \item \textbf{\textit{Remover}} ou \textbf{\textit{Dequeue}}
\end{enumerate}

O método \textbf{\textit{Inserir}} adiciona um novo elemento a estrutura de dados fazendo com que o mesmo seja representado como estando no fim da fila. Caso não exista outro elemento na fila, o mesmo é marcado tambem como estando no início da fila e portando sendo o próximo a ser removido pelo método  \textbf{\textit{Remover}}. 

O metodo \textbf{\textit{Remover}} retira o elemento atualmente no início da fila, o elemento subsequente passa a ser o novo início e caso não existam novos elementos a mesma ficará vazia.

\chapter{Trabalhos correlatos}

Por se tratarem de elementos bem difundidos no ramo científico e de implementação simples, não foram encontrados trabalhos ou pesquisas relacionadas diretamente ao estudo das estruturas de pilha e fila, entretanto as mesmas são base base a implementação de uma enorme quantidade de algoritmos.

\chapter{Proposta}

\section{Implementação de Pilha}

Para a implementação da estrutura de Pilha na linguagem C++, a clase de nome Pilha foi criada, com sua estrutura definida no arquivo \textbf{\textit{Pilha.hpp}} e sua implementação definida no arquivo \textbf{\textit{Pilha.cpp}}

Os elementos da estrutura de dados serão inseridos no vetor de nome  \textbf{\textit{memória}} que foi definido com o valor inicial de 1000 pela constante \textbf{\textit{SIZE}} 

\begin{myenv}{Pilha.hpp}
\begin{lstlisting}
//
//  Pilha.hpp
//  computacao.cientifica.algoritmos.pilha
//
//  Created by Rafael Thomas Salgado on 07/06/17.
//  Copyright 2017 Rafael Thomas Salgado. All rights reserved.
//

#ifndef Pilha_hpp
#define Pilha_hpp
#define SIZE 1000


#include <stdio.h>
#include <string>
#include <iostream>

#endif /* Pilha_hpp */

using namespace std;

class Pilha {
private:
    string memoria[SIZE];
    int posicaoTopo;
    
public:
    Pilha();
    void empilha(string entrada);
    string desempilha();
    string topo();
    int qtdElementos();
};
\end{lstlisting}
\end{myenv}


\begin{myenv}{Pilha.cpp}
\begin{lstlisting}
//
//  Pilha.cpp
//  computacao.cientifica.algoritmos.pilha
//
//  Created by Rafael Thomas Salgado on 07/06/17.
//  Copyright 2017 Rafael Thomas Salgado. All rights reserved.
//

#include "Pilha.hpp"

Pilha::Pilha(){
    posicaoTopo = 0;
}

void Pilha::empilha(string entrada){
    if(posicaoTopo < SIZE){
        memoria[posicaoTopo] = entrada;
        posicaoTopo++;
        cout << "Elemento \"" << entrada << "\" empilhado com sucesso\n";
    } else {
        cout << "A Pilha esta cheia\n";
    }
    return;
}

string Pilha::desempilha(){
    
    if(posicaoTopo == 0){
        cout << "A Pilha esta vazia\n";
        return "";
    } else {
        posicaoTopo--;
        string aux = memoria[posicaoTopo];
        memoria[posicaoTopo]="";
        return aux;
    }
}

string Pilha::topo(){
    if(posicaoTopo == 0){
        cout << "A Pilha esta vazia\n";
        return "";
    } else {
        return memoria[posicaoTopo - 1];
    }
}

int Pilha::qtdElementos(){
    return posicaoTopo;
}
\end{lstlisting}
\end{myenv}



\section{Implementação de Fila}

\chapter{Resultados}

Nunc molestie nunc ac lorem dictum, sit amet pl

\chapter{Conclusão}

Nunc molestie nunc ac lorem dictum, sit amet pl

\printbibliography

\printindex

\end{document}
